% Following workaround required in order to get auctex to parse this file
% and update auto/macros.el properly.
\iffalse
\documentclass{article}
\begin{document}
\fi

\usepackage{etex}
\usepackage{etoolbox}
\usepackage{amsmath}
\usepackage{amssymb}
\usepackage{array}
\usepackage{calc}
\usepackage{catchfilebetweentags}
\usepackage[mathscr]{euscript}
\usepackage[bottom,hang,flushmargin]{footmisc}
\usepackage{graphicx}
\usepackage{listings}
\usepackage{mathtools}
\usepackage{pdfpages}
\usepackage{pstricks,pst-node}
\usepackage{soul}
\usepackage{verbatim}

%-- ENUMERATE --------------------------------------------------------------
%\usepackage{enumerate} % to change label style

%\usepackage[newenum,olditem]{paralist}  % to change label style, and for \begin{inparaenum}[]

\makeatletter
\@ifclassloaded{beamer}{%
  %\usepackage{hyperref}
  \usepackage{multimedia}
  \input{mytikz.tex}

  %\hypersetup{colorlinks=true,linkcolor=white,urlcolor=blue}

  % http://tex.stackexchange.com/questions/57441/how-to-include-existing-pdf-slides-into-my-beamer
  \newcommand*{\loadpresentation}[1]{{\beamer@inlecturefalse\input{#1}}}

  % http://tex.stackexchange.com/questions/87249/adjust-vertical-space-before-enumerated-list-in-beamer
  \newcommand{\setlistspacing}[3]{%
    \def\@ld{#1}
    \expandafter
    \def\csname @list\romannumeral\@ld \endcsname
    {\leftmargin\csname leftmargin\romannumeral\@ld \endcsname
      \topsep    #2
      \parsep    0\p@   \@plus\p@
      \itemsep   #3
    }%
  }
  %
  \newcommand{\anchorpoint}[1]{\tikz[overlay,remember picture] \node (#1) {};}
  %
  \newcommand*{\golla}[4][red]{% [color]{location}{width}{height}
    \begin{tikzpicture}[overlay,remember picture,thick]
      \node () at #2 [shape=ellipse,draw=#1,text height=#4] {~\hspace{#3}~} ;
    \end{tikzpicture}
  }
  %
  \newcommand*{\namedgolla}[5][red]{% [color]{location}{width}{height}{nodename}
    \begin{tikzpicture}[overlay,remember picture,thick]
      \node (#5) at #2 [shape=ellipse,draw=#1,text height=#4] {~\hspace{#3}~} ;
    \end{tikzpicture}
  }
  %
  \newcommand*{\floatingnote}[5][1-]{%
    \begin{tikzpicture}[remember picture,overlay]
      \pgftransformshift{\pgfpointanchor{current page}{center}}
      \path<#1> #2
      node[rectangle,rounded corners=4pt,fill=#4,opacity=.5]
      {\parbox{#3}{\raggedright #5}};
    \end{tikzpicture}
  }
  %
  \newcommand*{\popupnote}[4][1-]{% [overlay spec]{anchor}{(x,y)}{text}
    \begin{tikzpicture}[remember picture,overlay]
      \pgftransformshift{\pgfpointanchor{current page}{north west}}
      \path<#1> (#2) ++#3
      node[anchor=west,rectangle callout,rounded
      corners=4pt,fill=black!30,opacity=.5,callout absolute  pointer={(#2)}] {#4};
    \end{tikzpicture}
  }
  %
  \newcommand{\reference}[1] {%
    \begin{tikzpicture}[remember picture,overlay]
      \pgftransformshift{\pgfpointanchor{current page}{north}}
      \draw (0,-1) node[anchor=north] {\fbox{\smaller{Reference: #1}}};
    \end{tikzpicture}
  }
  %
  \newcommand*{\titleslide}[1] {%
    \begin{frame}{} \cl{\Large \textit{\textbf{#1}}} \end{frame}
    %\begin{frame}[plain]{} \cl{\Large \textit{{#1}}} \end{frame}}
  }
}{%
  \usepackage[inline,shortlabels]{enumitem}% /home/mandar/latex/enumitem.pdf
  \usepackage{url}
}
\makeatother
%  \b...{enumerate}[topsep=,partopsep=,parsep=,itemsep=] changes vert. sp.
%  \b...{enumerate*} produces inline environment
%  \b...{enumerate}[i)] changes label style (see package option shortlabels)
%---------------------------------------------------------------------------

%-- COLOURS ----------------------------------------------------------------
\providecommand{\amber}{}
\providecommand{\blue}{}
\providecommand{\brown}{}
\providecommand{\green}{}
\providecommand{\grey}{}
\providecommand{\red}{}
\providecommand{\white}{}

\renewcommand{\amber}[1] {\textcolor[rgb]{1.00,0.50,0.00}{#1}}
\renewcommand{\blue}[1] {\textcolor{blue}{#1}}
\renewcommand{\brown}[1] {\textcolor{brown}{#1}}
\renewcommand{\green}[1] {\textcolor[rgb]{0,0.7,0}{#1}}
\renewcommand{\grey}[1] {\textcolor{gray}{#1}}
\renewcommand{\red}[1] {\textcolor{red}{#1}}
\renewcommand{\white}[1] {\textcolor{white}{#1}}


%-- TABULARS ---------------------------------------------------------------
% tabularx:
% - total width of table has to be specified and is fixed
% - all columns of equal width; relatively hard to change column widths
% = C, L, R column types have to be defined
% + relatively easy to change vertical alignment globally (cf. cheatsheet)
% \usepackage{tabularx}
% \newcolumntype{C}{>{\centering\arraybackslash}X}
% \newcolumntype{L}{>{\raggedright\arraybackslash}X}
% \newcolumntype{R}{>{\raggedleft\arraybackslash}X}

% tabulary:
% + total width of table is indicative and a maximum
% + relatively better column width allocation
% = C, L, R column types pre-defined
% - vertical alignment has to be changed per cell using \parbox (cf. cheatsheet)
% \usepackage{tabulary}

% ctable (includes array, booktabs and tabularx):
% + combines advantages of tabularx, tabulary
\makeatletter
\@ifundefined{htlatex}{%
  \usepackage{ctable}
  \setupctable{%
    botcap, % topcap or sidecap
    captionskip=2pt, % space between caption and table
    mincapwidth=2in, % minimum width of caption
  }
  % Useful for colouring table columns in the presence of midrules
  % http://tex.stackexchange.com/questions/11198/coloring-columns-in-a-table-with-colortbl-and-booktabs
  \setlength{\aboverulesep}{0pt} % remove space above rules
  \setlength{\belowrulesep}{0pt} % remove space below rules
  \setlength{\extrarowheight}{.75ex} % add back space in body of the row
}
\makeatother

% tabu:
\usepackage[delarray]{tabu}

\usepackage{colortbl}

%---------------------------------------------------------------------------

%-- LISTINGS ---------------------------------------------------------------%
\lstset{basicstyle=\ttfamily,
  breaklines=true,        
  breakatwhitespace=true,
  columns=fullflexible,
  keepspaces=true,
  numbers=left,
  numbersep=12pt,
  numberstyle=\tiny,
  showstringspaces=false,
  xleftmargin=1cm,
  keywordstyle=\color{blue}\ttfamily,%\color[rgb]{0.1,0.2,0.9}\ttfamily,
  stringstyle=\color{red}\ttfamily,
  commentstyle=\color[rgb]{0,0.7,0}\ttfamily,
  morecomment=[l][\color{brown}]{\#}}
%---------------------------------------------------------------------------%

\newcommand{\bskip} {\bigskip}
\newcommand{\hugeskip} {\bigskip \bigskip}
\renewcommand{\mskip} {\medskip}
\newcommand{\sskip} {\smallskip}
\newcommand{\parab}[1] {\bigskip \noindent \textbf{#1}~}
\newcommand{\param}[1] {\medskip \noindent \textbf{#1}~}
\newcommand{\paras}[1] {\smallskip \noindent \textbf{#1}~}

\newcommand{\doublespacing} {\linespread{1.6}\selectfont}
\newcommand{\onehalfspacing} {\linespread{1.3}\selectfont}
\newcommand{\singlespacing} {\linespread{1.0}\selectfont}
\newcommand{\setspacing}[1] {\linespread{#1}\selectfont}
\def\spacing {\setspacing}

\newlength{\parindentincr}
\newcommand{\addindent} {\addtolength{\parindent}{5mm}}
\newcommand{\indentby}[1] {
  \settowidth{\parindentincr}{#1}
  \addtolength{\parindent}{\parindentincr}}
\newcommand{\deindent} {\addtolength{\parindent}{-5mm}}
\newcommand{\deindentby}[1] {
  \settowidth{\parindentincr}{#1}
  \addtolength{\parindent}{-\parindentincr}}

\newcommand{\blank}[1] {\rule{#1}{0.5pt}}
\newcommand{\blanklines}[1] {%
  \newcounter{linei}
  \setcounter{linei}{0}
  \whileboolexpr%
    {test{\ifnumcomp{\value{linei}}{<}{#1}}}
    {\vspace{8mm}\hrulefill\par \addtocounter{linei}{1}}
  }
\newcommand{\cl} {\centerline}
\newcommand{\command}[1] {\textcolor{brown}{\texttt{#1}}}
\newcommand{\dash} {\blank}
\renewcommand{\extrarowsep}[1] {\tabulinesep=#1}
\newcommand{\half} {\ensuremath{\frac{1}{2}}}
\newcommand{\highlight}[2] {\noindent\colorbox{#1}{ % 1st argument = colour
    \parbox{\dimexpr\linewidth-2\fboxsep}{#2}}} % 2nd argument = text
\newcommand{\important}[1] {\centerline{\framebox{\textcolor{red}{#1}}}}
\newcommand{\intersection} {\cap}
\newcommand{\latex} {\LaTeX\ }
\newcommand{\lb} {\linebreak}
\renewcommand{\marks}[1] {\hspace*{\fill}{[#1]}}
\newcommand{\mc} {\multicolumn}
\newcommand{\mybox}[1] {\centerline{\framebox{#1}}}
\newcommand{\myorigin} {\qdisk(0,0){0.1}}
\newcommand{\mysignature} {\includegraphics[scale=0.2]{signature}}
\newcommand{\mystrut}[1] {\rule[- #1 * \real{0.3}]{0pt}{#1}}
\newcommand{\otoprule} {\midrule[\heavyrulewidth]}
\newcommand{\overbar} {\overline}
\newcommand{\partialinput}[2] {\ExecuteMetaData[#2]{#1}}
\newcommand{\pic} {\includegraphics}
\newcommand{\pipe} {~\ensuremath{\mid}~}
\newcommand{\pto} {\vfill\hspace*{\fill}\textsc{p.t.o.} \newpage}
\newcommand{\ra} {\ensuremath{\rightarrow}}
\newcommand{\Ra} {\ensuremath{\Rightarrow}}
\newcommand{\sbcb} {\;\!} % space before close brace
\newcommand{\tb} {\textbackslash}
\newcommand{\torf} {\hfill \textsc{true / false}}
\newcommand{\ttilde} {\textasciitilde}
\ifdefined\ul
  \renewcommand{\ul}[1] {\underline{#1}}
\else
  \newcommand{\ul}[1] {\underline{#1}}
\fi
\newcommand{\undertilde}[1] {\mbox{\Large \raisebox{-5mm}{$\displaystyle\stackrel{#1}{\text{\textasciitilde}}$}}}
\newcommand{\union} {\cup}
\newcommand{\xor} {\ensuremath{\oplus}}

\newfont{\helvetica}{phvr8t at 10pt}
\newfont{\helveticasmall}{phvr8t at 9pt}
\newfont{\helveticabig}{phvr8t at 22pt}
\newfont{\helveticao}{phvro8t at 11pt}

% These commands seem to change fontsize only, but not the spacing between
% lines (at least on prosper slides). 
% To change both, use:
% \fontsize{x}{y}\selectfont
% Documentation: 
% \fontsize{size}{skip}
%     Set font size. The first parameter is the font size to switch to;
%     the second is the `\baselineskip' to use. The unit of both
%     parameters defaults to pt. A rule of thumb is that the
%     baselineskip should be 1.2 times the font size.
\newcommand{\smaller}{\footnotesize}
\newcommand{\smallest}{\scriptsize}
\newcommand{\smallsc}[1] {{\footnotesize\textsc{#1}}}
\newcommand{\setfontsizeandspacing}[2] {\fontsize{#1}{#2}\selectfont}

% http://www.physics.wm.edu/~norman/latexhints/conditional_macros.html
\ifdefined\labelenumiii
  \renewcommand{\labelenumiii}{(\roman{enumiii})}
\else
  \newcommand{\labelenumiii}{(\roman{enumiii})}
\fi

%\renewcommand{\baselinestretch}{1.5}
%\renewcommand{\labelenumiv}{(\arabic{enumiv})}

%---------------------------------------------------------------------------%
% http://tex.stackexchange.com/questions/4889/input-only-part-of-a-file
% \partialinput{from-line}{to-line}{filename}
\makeatletter
\newread\pin@file
\newcounter{pinlineno}
\newcommand\pin@accu{}
\newcommand\pin@ext{pintmp}
% inputs #3, selecting only lines #1 to #2 (inclusive)
\newcommand*\dummypartialinput [3] {%
  \IfFileExists{#3}{%
    \openin\pin@file #3
    % skip lines 1 to #1 (exclusive)
    \setcounter{pinlineno}{1}
    \@whilenum\value{pinlineno}<#1 \do{%
      \read\pin@file to\pin@line
      \stepcounter{pinlineno}%
    }
    % prepare reading lines #1 to #2 inclusive
    \addtocounter{pinlineno}{-1}
    \let\pin@accu\empty
    \begingroup
    \endlinechar\newlinechar
    \@whilenum\value{pinlineno}<#2 \do{%
      % use safe catcodes provided by e-TeX's \readline
      \readline\pin@file to\pin@line
      \edef\pin@accu{\pin@accu\pin@line}%
      \stepcounter{pinlineno}%
    }
    \closein\pin@file
    \expandafter\endgroup
    \scantokens\expandafter{\pin@accu}%
  }{%
    \errmessage{File `#3' doesn't exist!}%
  }%
}
\makeatother
%---------------------------------------------------------------------------%

\iffalse
\end{document}
\fi
